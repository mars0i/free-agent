\documentclass[12pt]{article}

\usepackage{natbib}
\bibliographystyle{chicagourl}
\usepackage[letterpaper,left=1.25in,right=1.25in,top=1.0in,bottom=1.0in]{geometry}
\usepackage{xspace}
%\usepackage{listings}
% This tells the listings package that it's Lisp it will be formatting:
%\lstset{language=Lisp} % for uses of lstlistings below
%\lstset{basicstyle=\ttfamily}
% e.g.: \begin{lstlisting} ... \end{lstlisting}

\setlength{\parindent}{0in}  % Don't indent paragraphs
\setlength{\parskip}{2.5ex}  % Add space between paragraphs

\newcommand{\UseGaramond}[1]{%
	\usepackage[urw-garamond]{mathdesign}  % goes before fontenc, fontspec, etc.
	\usepackage[T1]{fontenc}
	\usepackage{fontspec,xltxtra,xunicode}
	\defaultfontfeatures{Mapping=tex-text} % ditto
	\setmainfont[Mapping=tex-text]{#1}}

\UseGaramond{GaramondNo8}
%\UseGaramond{EB Garamond}

\newcommand{\cmnt}[1]{}
\renewcommand{\cmnt}[1]{{\color{red}[#1]}}
\newcommand{\fn}[1]{\footnote{#1}}

\newcommand{\ie}{i.e.\@\xspace}
\newcommand{\eg}{e.g.\@\xspace}
\newcommand{\cf}{cf.\@\xspace}
\newcommand{\etc}{etc.\@\xspace}
\newcommand{\viz}{viz.\@\xspace}
\newcommand{\vs}{vs.\@\xspace}

\newcommand{\pr}{\mathsf{P}} % probability
\newcommand{\expct}{\mathsf{E}} % expectation

\begin{document}

{\large Notes on perception2 k-snipe learning algorithm in free-agent}\\
Marshall Abrams\hfill\today\\

In {\tt free-agent.perception2}, the function {\tt k-snipe-pref}
does this (from the docstring):
\begin{quote}\vspace{-2ex}
    Decides whether snipe eats mush, and updates the snipe's mush-pref
    in  response to the experience if so, returning a possibly updated
    snipe  along with a boolean indicating whether snipe is eating. 
    (Note that  the energy transfer resulting from eating will occur
    elsewhere, in  response to the boolean returned here.)
\end{quote}\vspace{-2ex}
More specifically, {\tt k-snipe-pref} decides to eat iff the sign of
the (scalar) sensory input is such that $\mbox{size sensory input} -
\mbox{midpoint between actual mushroom sizes}$ has the same as the
sign of the snipe's (noisy) mushroom preference {\tt mush-pref}.
(The midpoint is thus treated as an innate parameter, or one that was
learned  earlier, as is the distance between the two mushroom sizes
below.)  The idea is that a positive value for {\tt mush-pref} means
``I prefer large mushrooms," while a negative value means ``I prefer
small mushrooms." (That doesn't mean that an individual with a
positive preference will only eat large mushrooms, however, since the
sensory data is quite noisy.)

If the snipe does eat, this allows it to collect information about
whether the mushroom is nutritious or somewhat poisonous.  (This is
similar to what rats do, as I understand it, refusing to eat a kind of
food if eating it was followed by illness.)  When a k-snipe eats, 
{\tt k-snipe-pref} calculates a new value for its {\tt mush-pref} by
calling {\tt calc-k-pref}:
\begin{quote}\vspace{-2ex}
    Calculate a new mush-pref for a k-snipe.  Calculates an
    incremental change in mush-pref, and then adds the increment to
    mush-pref.  The core idea of the  increment calculation is that if
    the (somewhat random) appearance of a mushroom has a larger
    (smaller) value than the midpoint between the two actual mushroom
    sizes, and the mushroom's nutrition turns out to be positive, then
    that's a reason to think that larger (smaller) mushrooms are
    nutritious, and the opposite for negative nutritional values. 
    Thus it makes sense to calculate the increment as a scaled value
    of the product of the mushroom's  nutrition and the difference
    between the appearance and the midpoint.  Thus positive values of
    mush-pref mean that in the past large mushrooms have often be
    nutritious, while negative values mean that small mushrooms have
    more often been nutritious, on average.
\end{quote}\vspace{-2ex}
Specifically, {\tt calc-k-pref} adds the following increment to the
current value of {\tt mush-pref}:
\[
    \mbox{nutrition value of mushroom} \times  \frac{\mbox{sensory
    input} - \mbox{mushroom midpoint}}{\mbox{distance from low to high
    mushroom size}} \times \mbox{dt}
\]
where {\tt dt} is a small number, such as 0.001.  (The numerator of
the division above is the reciprocal of the {\tt mush-size-scale}
global parameter.)  The result of adding the above increment to {\tt
mush-pref} is then returned to {\tt k-snipe-pref}.  As explained in
the indented docstring above, the general  idea is that a positive
nutritional value pushes {\tt mush-pref} in the direction of the
sensory data's difference from the midpoint.

This algorithm is very roughly like a very simplified version of part of
the the algorithm that Feldman and Friston \citeyearpar[p.\
9]{FeldmanFriston:AttentionUncertainty} use to model the Posner
experimental setup\fn{
    F\&F's model was in fact my starting point in developing this simple
    method of learning mushroom preferences, even though what I ended up
    with is very different and much, much simpler.}
A series of eating experiences here takes on a role like the cue signal
in the Posner setup.  In F\&F's model, the result of the signal is due
to an interated calculation in response to it.  Here it's not a
persistent signal to which the system responds, but a series of
size-and-nutrition signals from different mushrooms. In F\&F's model,
the signal from the cue (along with signals from two other regions)
incrementally updates a hidden (vector-valued) variable $x$.  This
variable indirectly affects the perception of signals in the other regions by
affecting the precision which will scale the effects of those signals.  In
the perception2 mushroom preference algorithm, there's no direct analog
of precision, since {\tt mush-pref}, the closest analog of F\&F's $x$, is simply
multiplied by the adjusted input signal, with the result tested for a 
positive value, in order to decide whether to eat.  Scaling the size signal by
a precision at this point wouldn't have any effect on the sign of the result, so
there's no reason, in this kind of scheme, to do that.  Part of the reason
that the method can be so simple is just that we're only trying to determine
a Boolean value based on a very simple sensory signal.  We're not even trying
to determine a scalar value from a continuous range.


\bibliography{phil}
\end{document}
